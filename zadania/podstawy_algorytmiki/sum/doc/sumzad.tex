\documentclass{spiral-kurs}
\def\title{Suma liczb pierwszych}
\def\id{sum}
\def\TL{0.8~s}
\def\ML{256~MB}
\begin{document}
\makeheader
%
  
    Aby udowodnić, że potrafisz szybko znajdować liczby pierwsze, oblicz sumę liczb pierwszych znajdujących się w przedziale $[a, b]$, dla różnych zadanych $a$ i $b$.

    \section{Wejście}
    W pierwszym wierszu wejścia znajduje się liczba naturalna $T \leq 100\,000$, ilość przedziałów, które musisz rozpatrzyć. W kolejnych $T$ liniach znajdują się
    pary liczb naturalnych $(a_i, b_i)$ oddzielone spacją. Zawsze zachodzi $2 \leq a_i \leq b_i \leq 2\,000\,000$.
      
    \section{Wyjście}

    Dla każdej pary $(a_i, b_i)$ z wejścia podaj odpowiedź: sumę liczb pierwszych, które są większe lub równe $a_i$, a mniejsze lub równe $b_i$.

    
    \section{Wskazówki}
    Nie próbuj za każdym razem sprawdzać wszystkich liczb z każdego przedziału, ani sumować ich za pomocą pętli -- taki program działałby za wolno.
    Istotna wiedza potrzebna do rozwiązania zadania znajduje się w lekcji {\it Złożoność obliczeniowa}. Zwróć też uwagę, że wynik sumowania może nie zmieścić się
    w zakresie typu {\tt int} -- potrzebna będzie zmienna typu {\tt long long}.

    \example{0}
  \end{document}
