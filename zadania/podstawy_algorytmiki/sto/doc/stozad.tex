\documentclass{spiral-kurs}

  \def\title{Stonks}
  \def\id{sto}
  %\def\contest{}
%  \def\desc{Prosta algorytmika / Maksima i podciągi w tablicach}
  \def\TL{5~s}
  \def\ML{128 MB}
  
\begin{document}
  \makeheader
  
      Bajtazar jest menedżerem średniego szczebla w pewnej korporacji. Jego najbliższym zadaniem jest przygotowanie prezentacji dla zarządu, z której będzie jasno wynikało, że akcje firmy cały czas idą w górę. Ale...czy na pewno idą? Na szczęście Bajtazar jest mistrzem kreatywnego dobierania danych, i po prostu wybierze odpowiedni okres czasu do prezentacji tak, aby w tym okresie nic nie zmąciło nie zepsuło dobrego samopoczucia kierownictwa.
      
      Mając ciąg kolejnych notowań firmy będących liczbami naturalnymi, wybierz jego możliwie najdłuższy fragment, w którym każdy element jest większy od poprzedniego.
      
      
  \section{Wejście}
      W pierwszym wierszu wejścia podana jest liczba notowań $n$ ($1 \leq n \leq 1\,000\,000$). W drugim wierszu znajduje się $n$ liczb całkowitych oddzielonych pojedynczymi odstępami -- kolejne notowania. Wszystkie te liczby są dodatnie i nie przekraczają $1\,000\,000\,000$.
      
      
  \section{Wyjście}
      Twój program powinien wypisać na wyjście jedną liczbę całkowitą -- największą możliwą długość rosnącego fragmentu ciągu.

  \example{0}
  

\end{document}
