\documentclass{spiral-kurs}
\def\title{Mijanka}
\def\id{mij}
\def\TL{5~s}
\def\ML{256~MB}
\begin{document}
\makeheader
%
    Przemek obserwuje ruch samochodów na drodze. Droga jest dwukierunkowa i łączy
    wschodnią część miasta z~zachodnią. Ponieważ Przemek stoi na wzgórzu,
    to widzi dokładne położenie wszystkich samochodów.

    Przemek zastanawia się, ile par samochodów się minie.
    Dwa samochody się miną, jeśli jadą w przeciwnych kierunkach i znajdą się w tym samym punkcie.
    Zakładamy, że samochody nie zawracają, nie wyprzedzają oraz wszystkie jadą prosto przed siebie.
    Przemek kończy obserwację, jeśli wszystkie samochody dotrą na koniec drogi w który zmierzają.

    \section{Wejście}
    W pierwszym wierszu wejścia znajduje się jedna liczba całkowita $n$ ($1 \leq n \leq 1\,000\,000$),
    oznaczająca liczbę wszystkich samochodów, które widzi Przemek.

    W drugim wierszu wejścia znajduje się $n$ liczb całkowitych $s_0, s_1, \ldots, s_{n-1}$ ($s_i \in \{0, 1\}$),
    oznaczających kierunki kolejnych samochodów, podawanych w kolejności z zachodu na wschód.
    Liczba $s_i$ oznacza kierunek jazdy samochodu $i$: $0$ --- na wschód, $1$ --- na zachód.

    \section{Wyjście}
    Pierwszy i jedyny wiersz wyjścia powinien zawierać jedną liczbę całkowitą,
    równą liczbie par samochodów, które będą się mijały.

    \example{0}

    \medskip
    \noindent
    \textbf{Wyjaśnienie do przykładu:} Mijające się pary: ($1, 2$), ($1, 4$), ($1, 5$), ($3, 4$), ($3, 5$).

  \end{document}
