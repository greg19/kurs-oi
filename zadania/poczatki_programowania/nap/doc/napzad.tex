\documentclass{spiral-kurs}
\def\title{Na przemian}
\def\id{nap}
\def\TL{1~s}
\def\ML{256~MB}
\begin{document}
\makeheader
%
Antek z Zuzią często bawią się w pisanie liczb.
Na kartce zapisują na przemian liczby całkowite -- raz Zuzia, raz Antek.
Napisz program, który wczyta ciąg liczb napisany przez dzieci i wypisze liczby,
które kolejno pisało każde z dzieci.

\section{Wejście}
Pierwszy wiersz wejścia zawiera jedną liczbę całkowitą $n$ ($2 \le n \le 100\,000$), oznaczającą długość ciągu
liczb napisanego przez dzieci.
Drugi wiersz zawiera $n$ liczb całkowitych z zakresu od $1$ do $1000$, oddzielonych spacjami,
oznaczających kolejne liczby wypisane przez dzieci.
Dzieci wypisywały liczby na przemian: pierwszą Zuzia, drugą Antek itd.

\section{Wyjście}
Twój program powinien wypisać dwa wiersze.
W pierwszym wierszu powinien znaleźć się ciąg liczb napisanych przez Zuzię,
a w drugim -- ciąg liczb napisanych przez Antka.
Liczby w wierszach należy rozdzielać spacjami.

    \example{0}


  \end{document}
