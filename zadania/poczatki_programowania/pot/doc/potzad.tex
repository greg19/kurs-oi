\documentclass{spiral-kurs}
\def\title{Potęgi dwójki}
\def\id{pot}
\def\TL{1~s}
\def\ML{256~MB}
\begin{document}
\makeheader
%
    Napisz program, który wypisze, w kolejności rosnącej, wszystkie
    potęgi dwójki nie większe od danej liczby naturalnej $n$.

    \section{Wejście}
    Wejście zawiera jedną liczbę naturalną $n$
    ($1 \le n \le 1\,000\,000\,000$).
      
    \section{Wyjście}
    W kolejnych wierszach wyjścia Twój program powinien wypisać kolejne
    potęgi liczby 2 nie większe od $n$ (o~wykładnikach całkowitych nieujemnych),
    w porządku rosnącym.

    \example{0}


  \end{document}
