\documentclass{spiral-kurs}
\def\title{Stół (*)}
\def\id{sto}
\def\TL{1~s}
\def\ML{128~MB}
\begin{document}
\makeheader
%
Bajtazar kupuje meble przez Internet.
Znalazł już ładny stół i zestaw krzeseł.
Teraz zastanawia się, ile krzeseł może kupić, tak aby wszystkie zmieściły się przy stole.

Stół ma prostokątny blat o wymiarach $A \times B$ centymetrów.
Z kolei siedzisko krzesła, patrząc z góry, to kwadrat o wymiarach $K \times K$ centymetrów.
Dalej będziemy traktować stół jako prostokąt, a krzesła -- jako kwadraty.

Nad jednym z brzegów siedziska (kwadratu) znajduje się oparcie.
Każde krzesło należy ustawić oparciem przy stole, tzn.\ brzeg z oparciem powinien pokrywać się z pewnym brzegiem stołu.
Ponadto siedzisko powinno \textbf{w całości} znajdować się pod blatem.
Oczywiście żadne dwa krzesła nie mogą na siebie nachodzić.
W naszych rozważaniach pomijamy nogi od stołu (możemy założyć, że są nieskończenie cienkie i znajdują się w rogach blatu).
Ile krzeseł zmieści się pod stołem?

\section{Wejście}
W jedynym wierszu wejścia znajdują się trzy liczby całkowite $A$, $B$ i $K$ ($1 \leq A,B,K \leq 500\,000\,000$)
oddzielone pojedynczymi odstępami,
oznaczające, odpowiednio, wymiary blatu stołu oraz wymiar siedziska krzesła.

\section{Wyjście}
Twój program powinien wypisać na wyjście maksymalną liczbę krzeseł, które zmieszczą się przy stole.

\section{Przykłady}
    \twocol{
      Dla danych wejściowych:

      \includefile{../in/\id0.in}

      poprawnym wynikiem jest:

      \includefile{../out/\id0.out}
    }{
      \includegraphics{\id rys}
    }


\medskip
\noindent
\textbf{Wyjaśnienie:}
Rysunek pokazuje przykładowe rozmieszczenie krzeseł przy stole.
Oparcia zostały zaznaczone pogrubionymi odcinkami.
Nie jest możliwe ustawienie jedenastu krzeseł.

      \bigskip
      \medskip
      \noindent
      Natomiast dla danych wejściowych:

      \includefile{../in/\id0a.in}

      \noindent
      poprawnym wynikiem jest:

      \includefile{../out/\id0a.out}


\end{document}
