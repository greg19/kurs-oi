\documentclass{spiral-kurs}
\usepackage[noend]{algorithmic}
\renewcommand\algorithmicthen{}
\renewcommand\algorithmicdo{}
\def\title{Czy się zatrzyma? (*)}
\def\id{czy}
\def\TL{1~s}
\def\ML{128~MB}
\begin{document}
\makeheader
%
Bajtazar przechadzał się koło Biblioteki Uniwersyteckiej w Warszawie i na jednej z fasad zobaczył
fragment programu opatrzony pytaniem ,,Czy się zatrzyma?''.
Problem wyglądał intrygująco, dlatego Bajtazar postanowił zająć się nim po powrocie do domu.
Niestety, gdy zapisywał kod na kartce (od razu zamieniając go na język C++), popełnił błąd i zanotował:

\smallskip
\begin{center}
\begin{minipage}{5cm}
\begin{algorithmic}
\WHILE {($i\ != 1$)}
 \IF{($i\ \%\ 2 == 0$)}

    \STATE $i\,=\,i\,/\,2$;
 \ELSE
    \STATE $i\,=\,3\,*\,i\,+\,3$;
  \ENDIF

\ENDWHILE
\end{algorithmic}
\end{minipage}
\end{center}
    \smallskip

\noindent
Bajtazar próbuje teraz ustalić, dla jakich wartości początkowych zmiennej $i$ zapisany przez niego program zatrzyma się.
Zakładamy przy tym, że zmienna $i$ ma nieograniczony rozmiar, tj.\ może przyjmować dowolnie duże wartości.

    \section{Wejście}
Pierwszy i jedyny wiersz wejścia zawiera jedną liczbę całkowitą $i$ ($2 \leq i \leq 10^{14}$), dla której
należy sprawdzić, czy podany program zatrzyma się.

    \section{Wyjście}
W pierwszym i jedynym wierszu wyjścia Twój program powinien wypisać jedno słowo \texttt{TAK}, jeśli
program zatrzyma się dla podanej wartości $i$, lub \texttt{NIE} w przeciwnym przypadku.

    \example{0}

  \end{document}
