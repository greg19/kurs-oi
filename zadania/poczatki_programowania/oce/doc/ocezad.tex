\documentclass{spiral-kurs}
\def\title{Oceny}
\def\id{oce}
\def\TL{1~s}
\def\ML{256~MB}
\begin{document}
\makeheader
%
  Na szkolnej wywiadówce Pani rozdała każdemu z rodziców listę wszystkich ocen, które ich dziecko dostało od początku edukacji.
  Mama Zdzisia, spojrzawszy na tę listę, stwierdziła, że ocen tych jest bardzo dużo i~na pierwszy rzut oka nie widać,
  czy Zdzisio jest dobrym uczniem, czy nie.
  Mama Zdzisia chciałaby wiedzieć dokładnie, ile Zdzisio ma jedynek, ile dwójek, ile trójek, ile czwórek, ile piątek, a ile szóstek.
  Ponieważ sama boi się, że pomyli się w liczeniu, poprosiła Cię o pomoc.
    
  \section{Wejście}
  W pierwszym wierszu wejścia znajduje się jedna liczba całkowita $n$ ($1 \le n \le 1000$) oznaczająca łączną liczbę ocen Zdzisia.
  W drugim wierszu znajduje się $n$ liczb ze zbioru $\{1,2,3,4,5,6\}$, oddzielonych spacjami,
  oznaczających kolejne oceny w karierze Zdzisia.
    
  \section{Wyjście}
  Twój program powinien wypisać sześć liczb oddzielonych spacjami, oznaczającyh kolejno:
  liczbę jedynek, liczbę dwójek, liczbę trójek, liczbę czwórek, liczbę piątek oraz liczbę szóstek uzyskanych przez Zdzisia.

    \example{0}


  \end{document}
