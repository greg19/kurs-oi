\documentclass{spiral-kurs}
\def\title{Król}
\def\id{kro}
\def\TL{1~s}
\def\ML{256~MB}
\begin{document}
\makeheader
%
    Na pewnym polu na szachownicy $8 \times 8$ stoi król.
    Chcielibyśmy dowiedzieć się, na ile różnych pól może przeskoczyć ta figura w jednym ruchu\footnote{
      Ruchy króla szachowego są zilustrowane np. na stronie \texttt{http://pl.wikipedia.org/wiki/Król\_(szachy)}.
    }.
    Zakładamy, że na szachownicy nie ma w tym momencie żadnych innych figur.

    \section{Wejście}
    Pierwszy i jedyny wiersz wejścia zawiera jedną literę $k$ oraz jedną cyfrę $w$,
    oddzielone spacją.
    Litera $k$ oznacza kolumnę szachownicy ($k \in \{\mathtt{a},\ldots,\mathtt{h}\}$),
    a cyfra $w$ oznacza wiersz szachownicy ($w \in \{1,\ldots,8\}$).
      
    \section{Wyjście}
    W jedynym wierszu wyjścia Twój program powinien wypisać jedną liczbę całkowitą
    -- liczbę pól szachownicy, na które może przeskoczyć król umieszczony na zadanym polu.

    \example{0}

    \noindent
    natomiast dla danych wejściowych:

    \includefile{../in/\id 0a.in}

    \noindent
    poprawnym wynikiem jest:
    
    \includefile{../out/\id 0a.out}



  \end{document}
