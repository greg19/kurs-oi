\documentclass{spiral-kurs}
\def\title{Wielkanoc (*)}
\def\id{wie}
\def\TL{1~s}
\def\ML{256~MB}
\begin{document}
\makeheader
%
    Zadaniem Twojego programu jest wyznaczyć datę Wielkanocy w danym roku kalendarzowym.
    Interesuje nas tylko XIX, XX, XXI i XXII wiek i kalendarz gregoriański.
    W kościołach zachodnich Wielkanoc przypada w~pierwszą niedzielę po pierwszej wiosennej pełni Księżyca, przypadającej po 21 marca.

    Metody wyznaczania daty Wielkanocy podali m.in.\ Gauss\footnote{\texttt{http://pl.wikipedia.org/wiki/Wielkanoc\#Dla\_kalendarza\_gregoria.C5.84skiego}}
    i Meeus\footnote{\texttt{http://pl.wikipedia.org/wiki/Wielkanoc\#Dla\_kalendarza\_gregoria.C5.84skiego\_2}}.

    \section{Wejście}
    Na wejściu znajduje się jedna liczba całkowita $r$
    ($1800 \le r < 2200$), oznaczająca rok.
      
    \section{Wyjście}
    Twój program powinien wypisać na wyjście dwie liczby całkowite
    $d$ i $m$, oddzielone spacją, oznaczające dzień (liczba między 1 a 31)
    i miesiąc (liczba między 1 a 12), w którym w roku $r$ obchodzona jest Wielkanoc.

    \example{0}

    \noindent
    natomiast dla danych wejściowych:

    \includefile{../in/\id 0a.in}

    \noindent
    poprawnym wynikiem jest:
    
    \includefile{../out/\id 0a.out}


  \end{document}
