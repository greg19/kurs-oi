\documentclass{spiral-kurs}
\def\title{Największy wspólny dzielnik}
\def\id{nwd}
\def\TL{1~s}
\def\ML{256~MB}
\begin{document}
\makeheader
%
    W tym zadaniu celem jest obliczyć, jaka jest największa dodatnia liczba całkowita,
    która dzieli podane liczby naturalne $a_1,\ldots,a_n$.

    \section{Wejście}
    W pierwszym wierszu wejścia znajduje się jedna liczba całkowita $n$
    ($2 \le n \le 1000$).
    W drugim wierszu znajduje się $n$ liczb całkowitych $a_1,\ldots,a_n$
    ($1 \le a_i \le 1000$), oddzielonych spacjami, oznaczających liczby,
    których NWD szukamy.
      
    \section{Wyjście}
    Twój program powinien wypisać jedną liczbę będącą największym wspólnym dzielnikiem
    liczb $a_1,\ldots,a_n$.

    \example{0}


  \end{document}
