\documentclass{spiral-kurs}
\def\title{Kody szesnastkowe}
\def\id{kod}
\def\TL{1~s}
\def\ML{256~MB}
\begin{document}
\makeheader
%
    Kody ASCII znaków reprezentuje się czasami w postaci \emph{szesnastkowej}.
    Taka reprezentacja jest dwucyfrową liczbą zapisaną w układzie szesnastkowym.
    W układzie szesnastkowym występują ,,cyfry'' od 0 do 15, przy czym pierwsze dziesięć
    cyfr oznacza się normalnie, a cyfry od 10 do 15 oznacza się odpowiednio literami od A do F.
    Aby przeliczyć kod szesnastkowy na kod ASCII znaku, mnożymy więc pierwszą cyfrę kodu
    (cyfrę ,,dziesiątek'', a dokładniej, szesnastek) przez 16 i dodajemy drugą cyfrę, czyli
    cyfrę jedności.
    Twoim zadaniem jest napisanie programu, który pozwoli na automatyczne przeliczanie kodów
    szesnastkowych na kody ASCII.

    \section{Wejście}
    Na wejściu znajdują się dwa znaki określające poprawny kod szesnastkowy
    znaku z kodu ASCII.
    Pierwszy znak jest cyfrą między \texttt{2} a \texttt{7}, a drugi -- cyfrą lub wielką literą
    między \texttt{A} a \texttt{F}.
      
    \section{Wyjście}
    W pierwszym wierszu Twój program powinien wypisać kod ASCII znaku,
    a w drugim -- sam znak.
    Możesz założyć, że kod szesnastkowy na wejściu nie będzie odpowiadał
    żadnemu znakowi specjalnemu, czyli takiemu, którego nie da się ładnie wypisać.

    \example{0}

  \end{document}
