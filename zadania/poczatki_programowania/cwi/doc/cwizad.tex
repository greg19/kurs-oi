\documentclass{spiral-kurs}
\def\title{Ćwiartka}
\def\id{cwi}
\def\TL{1~s}
\def\ML{256~MB}
\begin{document}
\makeheader
%
  Napisz program, który dla danego punktu na płaszczyźnie sprawdzi,
  w której ćwiartce układu współrzędnych się on znajduje.
  Może jednak być tak, że punkt nie znajduje się w żadnej ćwiartce --
  leży na jednej z osi lub w~środku układu współrzędnych.
  Wówczas program powinien to stwierdzić.

  \section{Wejście}
  Na wejściu znajdują się dwie liczby całkowite oddzielone spacją,
  $x$ i $y$ ($-1\,000\,000\,000 \le x,y \le 1\,000\,000\,000$), oznaczające
  współrzędne danego punktu.

  \section{Wyjście}
  Jeżeli podany punkt nie leży na żadnej z osi, Twój program powinien wypisać:
  \texttt{I}, \texttt{II}, \texttt{III} lub \texttt{IV},
  w przypadku gdy punkt należy do, odpowiednio, pierwszej, drugiej, trzeciej lub czwartej ćwiartki
  układu współrzędnych.
  Jeżeli punkt leży w środku układu współrzędnych, program powinien wypisać liczbę \texttt{0}.
  W przeciwnym razie, program powinien wypisać \texttt{OX} (duże \texttt{O} i duże \texttt{X}),
  jeśli punkt leży na osi X, a \texttt{OY} -- jeśli punkt leży na osi Y.

  \example{0}

  \noindent
  a dla danych wejściowych:

  \includefile{../in/\id0a.in}

  \noindent
  poprawnym wynikiem jest:

  \includefile{../out/\id0a.out}

  \noindent
  natomiast dla danych wejściowych:

  \includefile{../in/\id0b.in}

  \noindent
  poprawnym wynikiem jest:

  \includefile{../out/\id0b.out}


\end{document}
