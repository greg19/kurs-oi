\documentclass{spiral-kurs}
\def\title{Trójkąt}
\def\id{tro}
\def\TL{1~s}
\def\ML{256~MB}
\begin{document}
\makeheader
%
    Twoim zadaniem jest stwierdzić, czy z trzech odcinków o podanych długościach
    można zbudować trójkąt o~dodatnim polu.

    \section{Wejście}
    Pierwszy i jedyny wiersz wejścia zawiera trzy liczby naturalne $a$, $b$, $c$
    ($1 \le a, b, c \le 1\,000\,000\,000$), oddzielone spacjami.
    Liczby te oznaczają długości trzech odcinków.

    \section{Wyjście}
    Twój program powinien wypisać jedno słowo \texttt{TAK} lub \texttt{NIE},
    w zależności od tego, czy z odcinków o długościach takich jak na wejściu
    można zbudować niezdegenerowany trójkąt, czyli trójkąt o dodatnim polu.

    \example{0}

    \noindent
    a dla danych wejściowych:

    \includefile{../in/\id 0a.in}

    \noindent
    poprawnym wynikiem jest:
    
    \includefile{../out/\id 0a.out}

    \noindent
    oraz dla danych wejściowych:

    \includefile{../in/\id 0b.in}

    \noindent
    poprawnym wynikiem jest:
    
    \includefile{../out/\id 0b.out}


  \end{document}
