\documentclass{spiral-kurs}
\def\title{Pierwszy i ostatni}
\def\id{pie}
\def\TL{1~s}
\def\ML{256~MB}
\begin{document}
\makeheader
%
  Dany jest ciąg liczb całkowitych dodatnich określający ceny akcji danej spółki na giełdzie
  w kolejnych dniach (na koniec każdego dnia).
  Chcemy stwierdzić, czy któregoś dnia cena akcji wynosiła dokładnie $x$.
  Jeśli był więcej niż jeden taki dzień, Twój program powinien wyznaczyć
  pierwszy i ostatni taki dzień.

  \section{Wejście}
  Pierwszy wiersz wejścia zawiera dwie liczby całkowite $n$ oraz $x$ ($1 \le n \le 100\,000$, $1 \le x \le 10^9$),
  oddzielone spacją i oznaczające liczbę kolejnych dni, przez które notowano ceny akcji danej spółki,
  oraz interesującą nas cenę akcji.
  Drugi wiersz zawiera $n$ liczb całkowitych z zakresu od $1$ do $10^9$, oddzielonych spacjami,
  oznaczających ceny akcji spółki w kolejnych dniach.

  \section{Wyjście}
  Twój program powinien wypisać dwie liczby oddzielone spacją.
  Pierwszą z nich powinien być numer pierwszego dnia (będący liczbą między 1 a $n$),
  którego cena akcji wynosiła dokładnie $x$.
  Drugą natomiast powinien być numer ostatniego takiego dnia.

  Jeśli był tylko jeden dzień, gdy cena akcji wynosiła $x$, obie liczby na wyjściu powinny być takie same.
  Jeśli cena $x$ w ogóle nie wystąpiła, obie liczby powinny być równe $-1$.

    \example{0}

    \noindent
    a dla danych wejściowych:

    \includefile{../in/\id 0a.in}

    \noindent
    poprawnym wynikiem jest:
    
    \includefile{../out/\id 0a.out}

    \noindent
    natomiast dla danych wejściowych:

    \includefile{../in/\id 0b.in}

    \noindent
    poprawnym wynikiem jest:
    
    \includefile{../out/\id 0b.out}


  \end{document}
