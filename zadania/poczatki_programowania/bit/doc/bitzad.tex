\documentclass{spiral-kurs}
\def\title{Ciąg bitoniczny}
\def\id{bit}
\def\TL{2~s}
\def\ML{256~MB}
\begin{document}
\makeheader
%
Ciąg bitoniczny to ciąg, który najpierw rośnie, a potem maleje.
Dokładniej, w ciągu bitonicznym $a_1,\ldots,a_n$ istnieje takie $i \in \{1,\ldots,n\}$,
że ciąg $a_1,\ldots,a_i$ jest ściśle rosnący, a ciąg $a_i,\ldots,a_n$ jest ściśle malejący.
Napisz program, który stwierdzi, czy ciąg podany na wejściu jest bitoniczny.
Zauważ, że w szczególności ciąg (ściśle) rosnący oraz ciąg (ściśle) malejący są ciągami bitonicznymi.

\section{Wejście}
W pierwszym wierszu wejścia znajduje się jedna liczba naturalna $n$ ($1\le n\le 500\,000$),
oznaczająca długość ciągu.
W drugim wierszu znajduje się $n$ liczb całkowitych z zakresu od $1$ do $1\,000\,000$,
oznaczających kolejne elementy ciągu.

\section{Wyjście}
Twój program powinien wypisać na wyjście jedno słowo \texttt{TAK} lub \texttt{NIE},
oznaczające, czy ciąg podany na wejściu jest bitoniczny.

  \example{0}

  \noindent
  i dla danych wejściowych:

  \includefile{../in/\id0a.in}

  \noindent
  poprawnym wynikiem jest:

  \includefile{../out/\id0a.out}

  \noindent
  natomiast dla danych wejściowych:

  \includefile{../in/\id0b.in}

  \noindent
  poprawnym wynikiem jest:

  \includefile{../out/\id0b.out}

  \noindent
  oraz dla danych wejściowych:

  \includefile{../in/\id0c.in}

  \noindent
  poprawnym wynikiem jest:

  \includefile{../out/\id0c.out}


\end{document}
