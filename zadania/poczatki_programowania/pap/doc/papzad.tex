\documentclass{spiral-kurs}
\def\title{Papryczki logarytmiczne (*)}
\def\id{pap}
\def\TL{1~s}
\def\ML{256~MB}
\begin{document}
\makeheader
%
  Hitem sezonu w bajtockim warzywniaku są papryczki logarytmiczne.
  Jak sama nazwa wskazuje, wagi papryczek, wyrażone w gramach, są wyłącznie
  potęgami dwójki między $2^0$ a $2^k$.

  Typowy przebieg transakcji w warzywniaku wygląda tak, że klient prosi
  o $x$ gramów papryczek i sprzedawca wydaje mu dokładnie taką ich masę,
  posługując się wyłącznie całymi papryczkami.
  Może się niestety tak zdarzyć, że zamówienie klienta będzie niemożliwe
  do spełnienia.
  Pomóż sprzedawcy sprawdzić, na ile jest on zabezpieczony przed taką feralną
  sytuacją, czyli wyznaczyć najmniejszą taką liczbę naturalną $x$ z zamówienia
  nie do zrealizowania.

\section{Wejście}
  Pierwszy wiersz wejścia zawiera jedną liczbę całkowitą $k$
  ($1\le k\le 10$), oznaczającą, że masy papryczek znajdujących się na stanie
  warzywniaka to $2^0,2^1,\dots,2^k$.
  Drugi wiersz zawiera $k+1$ liczb całkowitych dodatnich $p_0,p_1,\dots,p_k$
  nie większych niż $1000$, pooddzielanych spacjami
  i oznaczających dokładny stan sklepu: $p_0$ papryczek o wadze $1$, $p_1$ papryczek
  o wadze $2$, \ldots, $p_k$ papryczek o wadze $2^k$.

\section{Wyjście}
   Pierwszy i jedyny wiersz wyjścia powinien zawierać jedną liczbę całkowitą
   dodatnią $x$ -- najmniejszą wartość zamówienia, której sprzedawca nie
   będzie w stanie zrealizować.

   \example{0}

   \medskip
   \noindent
   \textbf{Wyjaśnienie do przykładu:} Wszystkie wartości $x$ od $1$ do $8$
   można osiągnąć przy aktualnym stanie sklepu; oto przykładowe takie
   przedstawienia:
   $1=1$, $2=1+1$, $3=1+2$, $4=4$, $5=1+4$, $6=1+1+4$, $7=1+2+4$, $8=1+1+2+4$.
   Wartości $x=9$ oczywiście nie da się tak osiągnąć.

 \end{document}
